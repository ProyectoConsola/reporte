\documentclass[letterpaper]{article}
\usepackage[top=2cm, bottom=2cm, left=1cm, right=1cm]{geometry}
\usepackage[spanish]{babel}
\usepackage{graphicx}
\usepackage{hyperref}
\usepackage{caption}
\usepackage{multicol}

\graphicspath{ {./images} }

\setlength{\columnsep}{1cm}

\title{
    {\Huge Aplicación de ingeniería para la creación de una Videoconsola}\\
    {\LARGE Demostración práctica de conocimientos de ingeniería en
    Computación, Electrónica y Eléctrica}
}
\author{
    Aaron Uriel Guzman Cardoso\\
    \texttt{2100554h@umich.mx}
    \and
    Artemisa Cortes Jacobo\\
    \texttt{1825812h@umich.mx}
    \and
    Brandon Alejandro Piñón Estanislado\\
    \texttt{2100730h@umich.mx}
    \and
    Carlos Edwin Bautista Zintzun\\
    \texttt{2105732g@umich.mx}
    \and
    Uriel Felipe Vazquez Orozco\\
    \texttt{1825946g@umich.mx}
    \and
    Maria Fernanda Mendez Sanchez\\
    \texttt{2100632c@umich.mx}
    \and
    Fernanda Montserrado Fabian Reyes\\
    \texttt{2100721a@umich.mx}
    \and
    Jorge Correa Gutierrez\\
    \texttt{2100536a@umich.mx}
}
\date{\today}

\begin{document}

\maketitle

\begin{multicols}{2}

\begin{abstract}
    Se introduce al lector detalles de la propuesta de una creación de
    consola, se aborda por qué pensamos que es útil para enseñar a estudiantes
    más jóvenes y comentamos el proceso de ingeniería necesario para su
    correcto desarrollo.\\
    Se toma en consideración la necesidad de dejar algo de provecho a quien
    interactúe con el producto, el usuario final podrá ser capaz de aplicar
    modificaciones simples en el producto final y tener conocimiento
    superficial del desarrollo del producto final.
\end{abstract}

\textbf{\textit{Palabras clave—}} Desarrollo de software, Ingeniería Electrónica,
Ingeniería en Computación, Consola DIY.

\section{Introducción}
Entre las personas generalmente existe la duda de lo que realmente es una
ingeniería, pues se suele confundir el trabajo de un técnico con el de un
ingeniero, ignorando aspectos de resolución de problemas y creación de nuevos
productos, que resulta ser parte importante de la ingeniería. Se plantea dar
a conocer el proceso que se realiza para resolver problemas por medio de
ingeniería con la creación de una consola de videojuegos.\\
El desarrollo de una consola de videojuegos es un proceso extenso que
requiere de la aplicación de múltiples disciplinas de ingeniería por lo que
es una excelente demostración de la aplicación de ingeniería. Para tener un
resultado final satisfactorio es necesario tener definidos requisitos y
expectativas, alcance, presupuesto y metodología de trabajo. Estos
aspectos toman en cuenta el tiempo disponible, la experiencia y disciplina del
equipo.\\
Este proyecto será expuesto en el tianguis de la ciencia los días viernes
19 de abril de 8:00 a 16:00 y sábado 20 de abril de 10:00 a 17:00 en
colaboración con los proyectos de ``'' y ``'' por lo que será necesario tener
tablones contiguos para una correcta complementación de los proyectos.

\subsection{Objetivo}
Dar una mejor noción a estudiantes sobre el proceso de ingeniería para la
resolución de problemas a través del desarrollo de una consola con juegos
simples que cumplan con ser divertidos.\\
Se desarrollará un proyecto capaz de tener modificaciones simples en los
aspectos de software por parte del usuario final.\\
Este proyecto representará principalmente los aspectos de aplicación de
ingeniería en Computación e ingeniería Electrónica por lo que será documentado
el proceso de diseño de software y hardware de manera fácil de entender para
personas agenas a esta disciplina, pensando en estudiantes jóvenes que todavía
no empiezan su carrera universitaria.

\section{Especificación}
La consola presentará medios para la generación de audio y video junto con
una implementación de entrada para hasta dos jugadores, todo usando
tecnologías analógicas y digitales. Se realizará retroalimentación al usuario
usando estos medios.\\
Se tendrá disponibles tres videojuegos principales, podrán ser escogidos de
manera aleatoria dentro de un menú principal y estarán adaptados para
ejecutarse con un tiempo de juego de tres minutos o menos. Los juegos
programados para la consola serán los siguientes:
\begin{enumerate}
    \item \textbf{Menú principal}. Menú con acceso a todos los juegos,
        opcionalmente se podrá activar la dinámica de escogerlos
        aleatoriamente.
    \item \textbf{Pong}. Un juego simple de dos jugadores en el que se simula
        una partida de tenis de mesa.
    \item \textbf{Space Invaders}. Juego arcade donde se eliminan oleadas de
        enemigos que se acercan cade vez más al jugador.
    \item \textbf{Complemento de bicicleta.} Se mostrará un mapa bastante
        simplificado de la universidad en donde tendrás que desplazarte de un
        punto A a un punto B, a través de pedalear en una bicicleta real. Solo
        se podrán realizar giros en intersecciones.
\end{enumerate}
Los aspectos modificables de los juegos por parte del usuario podrán ser: 
que tan rápido se mueven, cantidad y comportamiento de ciertos elementos en
pantalla, y dificultad.

\section{Ingeniería}
Es necesario la creación de documentación relacionada al desarrollo incluyendo
investigaciones necesarias para la correcta realización de una actividad y las
herramientas utilizadas para dicha actividad.\\
Este proyecto hará uso de microcontroladores así como todos sus puertos y
sensores disponibles para la generación de video y audio. El comportamiento de
dicho microcontrolador será dictado por software diseñado específicamente para
él por medio de lenguajes de programación.\\
Todo estará en un diseño hecho específicamente para el proyecto con un diseño
que asemeje una Arcade pequeña. Dentro estará el diseño del circuito soldado
final junto con una pantalla y altavoces.

\end{multicols}

\end{document}
