\documentclass[letterpaper]{article}
\usepackage[top=2cm, bottom=2cm, left=1cm, right=1cm]{geometry}
\usepackage[spanish]{babel}
\usepackage{graphicx}
\usepackage{hyperref}
\usepackage{caption}
\usepackage{multicol}

\graphicspath{ {./images} }

\setlength{\columnsep}{1cm}


\begin{document}

\title{
    {\Huge Creación de una consola de videojuegos}\\
    {\LARGE Una demostración práctica de conocimientos de ingeniería}
}
\author{
    {\large Borrador por Aaron Uriel Guzman Cardoso}\\
    {\large Facultad de Ingeniería Eléctica}\\
    {\large Universidad Michoacana San Nicolás de Hidalgo}\\
}
\date{Tianguis de la Ciencia, aprox. 21 de abril de 2024}
\maketitle

\newpage
\begin{multicols}{2}

\begin{abstract}
    En este artículo se documenta el proceso de ingeniería necesario para
    diseñar una consola de videojuegos simple a partir de un entorno de hardware
    bajos recursos, tomando en cuenta las necesidades de una consola: incluir
    uno o más juegos y cumplir con ser entretenida.\\
    Se toma en consideración la necesidad de dejar algo de provecho a quien
    interactúe con el producto final y que el proceso de desarrollo quede claro
    para quien lea este artículo.
\end{abstract}

\textbf{\textit{Palabras clave—}} Desarrollo de software, Ingeniería Electrónica,
Ingeniería en Computación, Consola DIY.

\section{Introducción}
Entre las personas generalmente existe la duda de lo que realmente es una
ingeniería, pues se suele confundir el trabajo de un técnico con el de un
ingeniero, ignorando la resolución de problemas, que resulta ser parte
importante de la ingeniería. Se plantea dar a conocer el proceso que se realiza
para resolver problemas por medio de ingeniería con la creación de una consola
de videojuegos.\\
El desarrollo de una consola de videojuegos es un proceso extenso que
requiere de la aplicación de múltiples disciplinas de ingeniería por lo que
es una excelente demostración de la aplicación de ingeniería. Para tener un
resultado final satisfactorio es necesario tener definidos requisitos y
expectativas, alcance, presupuesto y metodología de trabajo. Estos
aspectos toman en cuenta el tiempo disponible, la experiencia y disciplina del
equipo.\\
Este proyecto será expuesto durante el tianguis de la ciencia de 2024 en la
Universidad Michoacana San Nicolás de Hidalgo.

\subsection{Objetivo}
Dar una mejor noción a estudiantes sobre el proceso de ingeniería para la
resolución de problemas a través del desarrollo de una consola con juegos
que sean sencillos/simples pero cumplan con ser divertidos.

\section{Especificación}
\subsection{Alcance y tiempo disponible}
Al momento de escribir quedan aproximadamente cuatro meses para su presentación,
por lo es necesario tener especificaciones para un proyecto que se pueda
realizar en tres meses o menos, por lo que deberá de ser sencillo, simple y
seguir cumpliendo con la meta final.

\subsubsection{Hardware}
Se propone usar una PC de bajos recursos como base para la consola que disponga
de capacidades nativas de video, audio y gráficos acelerados por GPU, en este
caso una Raspberry Pi Zero 2 W. Con este equipo disponemos de un procesador de
1GHz de cuatro núcleos, 512 MB de RAM, conectividad Wifi y Bluetooth, puerto
micro-USB, salida de video HDMI así como 26 GPIOs con soporte a SPI, I2C y
UART.\\
Hay que tomar en cuanta algunas limitaciones que se tienen con la Raspberry Pi
Zero 2 W y algunas diferencias comparandola con microcontroladores que se pueden
solventar con componentes externos:
\begin{itemize}
    \item El alamcenamiento principal es una microSD y su máxima capacidad es de
        32 GB.
    \item El procesador es de 64-bits pero se ve limitado por la RAM, por lo que
        estamos limitados a ejecutar sistemas operativos de 32-bits.
    \item No incluye nativamente DACs ni ADCs.
    \item Solo tiene un puerto micro-USB.
\end{itemize}
La entrada del usuario podrá ser digital o analógica, y será através de
controles que imiten a los de una arcade o una NES, esta entrada será
digitalizada adecuadamente por un componente externo y recibida por la CPU
principal para ser interpretada.

\subsubsection{Sistema operativo}
Se propone un sistema operativo modificado para iniciar directamente a una
aplicación en pantalla completa que permita iniciar los juegos así como también
realizar ajustes en la consola.


\subsubsection{Juego}
Seguido a esto, se propone la creación de un único juego que aproveche las
capacidades del sistema, sea compatible con los controles diseñados y permita
dos jugadores simultaneos, el juego será un clon de "pong!".\\
El juego incluirá la posibilidad de guardar mejores puntajes, efectos de sonido
y gráficos simples en 2D.\\
Se propone programar usando C en conjunto con la biblioteca SDL usando el patrón
de desarrollo ECS.

\section{Desarrollo}
Aquí se incluirá todo lo relacionado al desarrollo del hardware y software
necesario.

\section{Resultados}
Se mencionará qué obtuvimos como producto final que se presentará en el tianguis
de la ciencia.

\section{Conclusiones}
Se incluirá lo aprendido tras el desarrollo. Como se trabajó con el equipo, que
metodología fue la más adecuada para trabajar, que dificultades hubo en el
desarrollo, etc.

\section{Bibliografía}
Otros artículos o libros que funcionaron como referencia para el desarrollo del
proyecto o que sea necesario referenciar en el texto del artículo.

\end{multicols}

\end{document}
